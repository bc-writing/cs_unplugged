%UTILE ? ajouter methi-ode au hasard mais avec bon deo-placement pourdiminuer SL(C) si possible !!!

\documentclass[a4,11pt]{article}
	\usepackage{template/css-color}
    \usepackage{template/css-gray}

    \usepackage{template/general}
    \usepackage{template/quote}
    \usepackage{template/algo}
    \usepackage{template/math}
    
    \usepackage[
        type={CC},
        modifier={by-nc-sa},
        version={4.0},
    ]{doclicense}
   

% WATERMARK
	\usepackage[printwatermark]{xwatermark}


% DRAFT MODE !
	\newwatermark[%
		allpages,%
		color=red!20,%
		angle=45,%
		scale=1.1,%
		xpos=-10, ypos=30]{Z! - BROUILLON - TEXTE NON SÛR - Z!}

	\newwatermark[%
		allpages,%
		color=red!20,%
		angle=45,%
		scale=1.1,%
		xpos=5, ypos=-5]{PRÉSENCE PROBABLE DE GROSSES ERREURS}


% NEED TO BE READ BY OTHERS !
%	\newwatermark[%
%		allpages,%
%		color=red!20,%
%		angle=45,%
%		scale=1.1,%
%		xpos=-10, ypos=30]{PRÉVERSION - TEXTE EN DEVENIR}
%
%	\newwatermark[%
%		allpages,%
%		color=red!20,%
%		angle=45,%
%		scale=1.1,%
%		xpos=5, ypos=-5]{BESOIN DE RELECTURES EXTERNES}


\begin{document}

\hspace{1cm}

\clearpage

%\addtitle{%
%    title      = {Baseball des couleurs - Une petite étude théorique},
%    name       = {Christophe BAL},
%    mail       = {projetmbc@gmail.com},
%    version    = {Version du 2017-01-21},
%    addlicence = yes,
%%    repo       = {https://github.com/bc-writings-algo-general/algorithms_with_paper},
%%    repofolder = {fr}
%}
%
%\vspace{3cm}
%
%\hrule
%
%\tableofcontents
%
%\vspace{4ex}
%
%\hrule
%
%\newpage
%
%
%
%\section{Origine du jeu}    % RELU. OK !
%    \input{content/origin.tex}
%
%
%\section{Les règles}    % RELU. OK !
%    \input{content/rules.tex}
%
%
%\section{Premier contact : la méthode \textit{\og une base à la fois \fg}}    % RELU. OK !
%    \input{content/algo_selection.tex}
%
%
%\section{Soyons opportuniste : la méthode \textit{\og on avance au mieux \fg}}    % RELU. OK !
%    \input{content/algo_bubble.tex}


\section{À la recherche d'une éventuelle solution optimale}    % RELU ? NON
	\subsection{Où tentons-nous d'aller ?}
    	A CREUSER :

	à cahque coup on va chercher à augmenter ou diminuer siuvant lsens de parcrousrs 

	par contre aux extrémités, on s'autoirsent à fair une boucle !



Il se trouve que la méthode \quote{on avance au mieux} n'est pas la plus efficace comme le montre l'exemple suivant où six mouvements totalement inutiles sont effectués si bien que l'on n'a toujours pas gagné à la neuvième étape.

\vspace{-0.4em}
\begin{multicols}{2}
	\begin{center}   % [4, 4, None, 0, 1, 1, 2, 2, 3, 3]
		\includegraphics[scale= 0.45]{content/optimal/wheredowego/algo_bubble/000.png}

		\includegraphics[scale= 0.45]{content/optimal/wheredowego/algo_bubble/001.png}

		\includegraphics[scale= 0.45]{content/optimal/wheredowego/algo_bubble/002.png}

		\includegraphics[scale= 0.45]{content/optimal/wheredowego/algo_bubble/003.png}

		\includegraphics[scale= 0.45]{content/optimal/wheredowego/algo_bubble/004.png}
	\end{center}

	\columnbreak
	\begin{center}   % [4, 4, None, 0, 1, 1, 2, 2, 3, 3]
		\includegraphics[scale= 0.45]{content/optimal/wheredowego/algo_bubble/005.png}

		\includegraphics[scale= 0.45]{content/optimal/wheredowego/algo_bubble/006.png}

		\includegraphics[scale= 0.45]{content/optimal/wheredowego/algo_bubble/007.png}

		\includegraphics[scale= 0.45]{content/optimal/wheredowego/algo_bubble/008.png}

		\includegraphics[scale= 0.45]{content/optimal/wheredowego/algo_bubble/009.png}
	\end{center}
\end{multicols}


\medskip

Or on peut gagner ici en seulement 9 coups ! Voici les mouvements à faire
\footnote{
	Notez au passage les enseignements que l'on peut tirer d'une configuration très, très particulière.
}.   DESSIN FAUX !!!!!!!!!!!!!

\vspace{-0.4em}
\begin{multicols}{2}
	\begin{center}   % [4, 4, None, 0, 1, 1, 2, 2, 3, 3]
		\includegraphics[scale= 0.45]{content/optimal/wheredowego/algo_bubble/000.png}

		\includegraphics[scale= 0.45]{content/optimal/wheredowego/algo_bubble/000.png}

		\includegraphics[scale= 0.45]{content/optimal/wheredowego/algo_bubble/000.png}

		\includegraphics[scale= 0.45]{content/optimal/wheredowego/algo_bubble/000.png}

		\includegraphics[scale= 0.45]{content/optimal/wheredowego/algo_bubble/000.png}
	\end{center}

	\columnbreak
	\begin{center}   % [4, 4, None, 0, 1, 1, 2, 2, 3, 3]
		\includegraphics[scale= 0.45]{content/optimal/wheredowego/algo_bubble/000.png}

		\includegraphics[scale= 0.45]{content/optimal/wheredowego/algo_bubble/000.png}

		\includegraphics[scale= 0.45]{content/optimal/wheredowego/algo_bubble/000.png}

		\includegraphics[scale= 0.45]{content/optimal/wheredowego/algo_bubble/000.png}

		\includegraphics[scale= 0.45]{content/optimal/wheredowego/algo_bubble/000.png}
	\end{center}
\end{multicols}


\medskip

Commençons par proposer une nouvelle modélisation du jeu qui soit bien plus fine que celles utilisées dans les deux sections précédentes. 
Que cherche-t-on à faire ? Trouver une solution peu coûteuse. Très bien ! Mais dans ce cas, comment évalue-t-on ce coup ? Nous choisissons de chercher à minimiser le nombre de déplacements du trou (donc toute opération autre que le déplacement d'un jeton ne sera pas comptabilisée).
Dans le premier cas ci-dessus, le coût est strictement plus grand que $9$, tandis que dans le second cas, il vaut exactement $9$. 


\medskip

Soit une configuration $\cal C$

??????????????????????????????????????????
notation en ligne car pratique mais ici on s'autorise le passage de la base tout à droite à celle tout à gauche
??????????????????????????????????????????

. Pour chaque jeton $j$, nous notons $deg \ j$ son degré d'éloignement qui est égal par définition aux nombres de bases entre sa base comprise et la base de sa couleur.
Par exemple, au début des exemples ci-dessus, tous les jetons ont un degré d'éloignement égal à un.


\medskip

Nous pouvons alors proposer la troisième méthode suivante (notez la concision de l'algorithme).   PAS BON PAS BON PAS BON PAS BON PAS BON PAS BON PAS BON PAS BON PAS BON PAS BON ( voir juste parès !!!!

\bigskip

\begin{algo}
	\Data{une configuration quelconque de début de jeu}
	\Result{une configuration où tous les jetons sont rentrées dans leur base}
	\vspace{0.4em}
    \Begin{
		\vspace{0.4em}
		\While{la configuration contient un jeton qui n'est pas dans sa base}{
			\vspace{0.4em}
			$j_1$ , $j_2$ , $j_3$ et $j_4$ désignent les jetons déplaçables.
			\\
			\vspace{0.4em}
			Pour $k$ allant de $1$ à $4$, on note $d_k$ est le dégré d'éloignement du jeton $j_k$ s'il était déplacé.
			\\
			\vspace{0.4em}
			Déplacer réellement un jeton $j_k$, éventuellement choisi au hasard, tel que $d_k$ soit minimal.
		}
    }
\end{algo}


\bigskip

Les instructions étant sans ambiguïté, nous allons pouvoir nous attaquer très sérieusement à la validation des propriétés de \quote{finitude} et de \quote{résolution}.

\begin{proof}
	????????????????????
	
	CODER POUR VOIR SI CEL MARCHE 
	
	PAR EXEMPLe, faire gaffe au cas ou les quatre jetons déplaçables ont le même degré dans la config du moment
	
	r r t n j j b b v v
	
	r r j n t j b b v v
	
	pas possible car boucle infinie possible
\end{proof}


	\subsection{Recherche de toutes les configurations possibles}

	\subsection{Analyse de l'algorithme donnant toutes les configurations possibles}


\section{Annexe : calculer le nombre de configurations possibles}    % BELLE CONNERIE !!! Tirage avec répétition !!!!!! Cf. multinome !!!
    On peut se demander combien de configurations de jeux sont possibles. Nous allons calculer ce nombre en cherchant directement une formule générale pour $n \geqslant 2$ bases. Dans ce cas nous avons au total $(2n - 1)$ jetons et $n$ \quote{couleurs} différentes.

\begin{enumerate}
	\item Tout d'abord, nous avons $n$ choix de bases où placer le trou.

	\item Une fois le trou placé dans une base ${\cal B}_0$, nous avons $n$ configurations possibles pour cette base ${\cal B}_0$ car seule la \quote{couleur} du jeton ajouté importe.

	\item Maintenant que la base ${\cal B}_0$ contient le trou et un jeton, que se passe-t-il pour les $(2n - 2)$ jetons restants et les $(n - 1)$ bases restantes ${\cal B}_1$ , ${\cal B}_2$ , \dots ,  ${\cal B}_{n - 1}$ ?
	
	\begin{enumerate}
		\item Pour la base ${\cal B}_1$, nous devons choisir deux jetons parmi $(2n - 2)$ sans tenir compte de l'ordre du tirage. Ce nombre est noté $\binom{2n - 2}{2}$ ce qui se lit \quote{$2$ parmi $(2n - 2)$}. Nous pouvons ici le calculer directement.
		
		En effet, nous avons $(2n - 2)$ choix pour le premier jeton $p$, et ensuite $(2n - 3)$ pour le deuxième jeton $d$. Si l'on tient compte de l'ordre cela fait $(2n - 2)(2n - 3)$ choix possibles. Or tirer $p$ puis $d$, ou $d$ puis $p$ ne change rien pour la configuration, donc le nombre de choix possibles est $\dfrac{(2n - 2)(2n - 3)}{2}$.
		
		\item De façon analogue, nous avons ensuite $\dfrac{(2n - 4)(2n - 5)}{2}$ pour la base ${\cal B}_2$.
		
		\item \dots \textit{etc.}
	\end{enumerate}
\end{enumerate}

Le nombre total $C(n)$ de configurations est donc 
\footnote{
	L'utilisation de points de suspension est un acte très critiquable mais il va nous permettre de faciliter la compréhension des formules manipulées.
}:
\begin{align*}
	C(n)
	  & = n
	      \times n 
	      \times \dfrac{(2n - 2)(2n - 3)}{2} 
	      \times \dfrac{(2n - 4)(2n - 5)}{2} 
	      \times \cdots 
	      \times \dfrac{4 \times 3}{2}
	      \times \dfrac{2 \times 1}{2}
	\\
	C(n)
	  & = \dfrac{n^2 \times (2n - 2)!}{2^{n-1}}
\end{align*}

Dans la seconde formule, le $2^{n-1}$ vient de ce que l'on divise par $2$ uniquement pour les bases ${\cal B}_1$ , ${\cal B}_2$ , \dots{} , et  ${\cal B}_{n - 1}$.
De plus, on utilise la factorielle d'un naturel non nul $k$ définie par $k! = 1 \times 2 \times 3 \times \cdots  \times k$.


\medskip

Ceci nous donne par exemple les valeurs suivantes (voir la remarque n°1 juste après).
	

\medskip

\begin{itemize}
	\item[\textbullet] $C(2) = 4$ ce qui est calculable directement en imaginant les configurations possibles avec deux bases.

	\medskip

	\item[\textbullet] $C(3) = 54$,
	                   $C(4) = 1440$ et
	                   $C(5) = 63\,000$.

	\medskip

	\item[\textbullet] $C(33) \approx 3,\!2 \times 10^{82}$ que l'on comparera à $10^{80}$ qui est une estimation du nombre d'atomes dans l'univers
	\footnote{
		Notons que si l'estimation du nombre d'atomes de l'univers est juste, alors il est tout simplement impossible \quote{d'écrire} tous les naturels de $1$ à $C(33)$ en associant chaque naturel à un seul atome de l'univers. Vertigineux !
	}.
\end{itemize}


\paragraph{Remarque n°1 :} \hspace{-1em} les calculs ont été faits via le logiciel SageMath utilisable directement en ligne à l'adresse suivante \url{https://sagecell.sagemath.org}. Le code utilisé est le suivant.

\bigskip

\begin{myverb}
def C(n):
    return n**2 * factorial(2*n - 2) / 2**(n - 1)

for k in range(2, 6):
    print C(k)

print C(33).n(10)

\end{myverb}


\medskip

\paragraph{Remarque n°2 :} \hspace{-1em} un argument de symétrie montre facilement que $C(n)$ est pair. Ceci se vérifie aussi à l'aide de la formule $C(n) = \dfrac{n^2 \times (2n - 2)!}{2^{n-1}}$.


\medskip

En effet, nous avons :
\begin{align*}
	C(n)
	  & = \dfrac{n^2 \times (2n - 2)!}{2^{n-1}}
	\\
	C(n)
	  & = n^2 \times \dfrac{(2n - 2) \times (2n - 3) \times (2n - 4) \times (2n - 5) \times \cdots \times 4 \times 3 \times 2 \times 1}{2 \times 2 \times \cdots \times 2 \times 2}
	\\
	C(n)
	  & = n^2 \times (n - 1) \times (2n - 3) \times (n - 2) \times (2n - 5) \times \cdots \times 2 \times 3 \times 1
\end{align*}


Comme $n$ ou $(n -1)$ est pair, nous retrouvons bien par le calcul que $C(n)$ est pair. 

\end{document}
